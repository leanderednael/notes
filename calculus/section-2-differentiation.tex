\section{Differentiation}

\subsection{Derivatives and Rates of Change}

\begin{paracol}{2}

\begin{tikzpicture}
\node [rounded-box] (box){\begin{minipage}{0.45\textwidth}
    The \textbf{tangent line} to the curve $y = f(x)$ at the point $P(a, f(a))$ is the line through $P$ with slope

    $$m = \lim_{x \rightarrow a} \frac{f(x) - f(a)}{x - a}$$

    provided that this limit exists.
\end{minipage}};
\node[rounded-box-title, left=10pt] at (box.north east) {Definition};
\end{tikzpicture}

\begin{tikzpicture}
\node [rounded-box] (box){\begin{minipage}{0.45\textwidth}
    The difference quotient

    $$\frac{\triangle y}{\triangle x} = \frac{f(x_2) - f(x_1)}{x_2 - x_1}$$

    is the \textbf{average rate of change} with respect to $x$ over the interval $[x_1, x_2]$.
\end{minipage}};
\node[rounded-box-title, left=10pt] at (box.north east) {Definition};
\end{tikzpicture}

\begin{tikzpicture}
\node [rounded-box] (box){\begin{minipage}{0.45\textwidth}
    The \textbf{instantaneous rate of change} is

    $$\lim_{\triangle x \rightarrow 0} \frac{\triangle y}{\triangle x} = \lim_{x_2 \rightarrow x_1} \frac{f(x_2) - f(x_1)}{x_2 - x_1}$$
\end{minipage}};
\node[rounded-box-title, left=10pt] at (box.north east) {Definition};
\end{tikzpicture}

\begin{tikzpicture}
\node [rounded-box] (box){\begin{minipage}{0.45\textwidth}
    The derivative $f'(a)$ is the instantaneous rate of change of $y = f(x)$ with respect to $x = a$.
\end{minipage}};
\node[rounded-box-title, left=10pt] at (box.north east) {Definition};
\end{tikzpicture}

\switchcolumn

\begin{tikzpicture}
\node [rounded-box] (box){\begin{minipage}{0.45\textwidth}
    The \textbf{derivative} of a function $f$ at a number $a$ is

    $$f'(a) = \lim_{h \rightarrow 0} \frac{f(a + h) - f(a)}{h}$$

    or equivalently

    $$f'(a) = \lim_{x \rightarrow a} \frac{f(x) - f(a)}{x - a}$$

    if this limit exists.
\end{minipage}};
\node[rounded-box-title, left=10pt] at (box.north east) {Definition};
\end{tikzpicture}

\begin{tikzpicture}
\node [rounded-box] (box){\begin{minipage}{0.45\textwidth}
    The derivative is defined as a function

    $$f'(x) = \lim_{h \rightarrow 0} \frac{f(x + h) - f(x)}{h}$$
\end{minipage}};
\node[rounded-box-title, left=10pt] at (box.north east) {Theorem};
\end{tikzpicture}

\begin{tikzpicture}
\node [rounded-box] (box){\begin{minipage}{0.45\textwidth}
    A function $f$ is \textbf{differentiable} at $a$ if $f'(a)$ exists. It is differentiable on an open interval $(a, b)$, $(a, \infty)$, $(-\infty, a)$ or $(-\infty, \infty)$ if it is differentiable at every number in the interval.
\end{minipage}};
\node[rounded-box-title, left=10pt] at (box.north east) {Definition};
\end{tikzpicture}

\begin{tikzpicture}
\node [rounded-box] (box){\begin{minipage}{0.45\textwidth}
    If $f$ is differentiable at $a$, then $f$ is continuous at $a$.
\end{minipage}};
\node[rounded-box-title, left=10pt] at (box.north east) {Theorem};
\end{tikzpicture}

\end{paracol}

\subsection{Linear Approximations and Differentials}

\begin{paracol}{2}

\begin{tikzpicture}
\node [rounded-box] (box){\begin{minipage}{0.45\textwidth}
    \textbf{Linearisation}:

    $$L(x) = f(a) + f'(a) (x - a)$$
\end{minipage}};
\node[rounded-box-title, left=10pt] at (box.north east) {Definition};
\end{tikzpicture}

\switchcolumn

\begin{tikzpicture}
\node [rounded-box] (box){\begin{minipage}{0.45\textwidth}
    \textbf{Differentials}:

    $$dy = df = f'(x) \, dx$$
\end{minipage}};
\node[rounded-box-title, left=10pt] at (box.north east) {Definition};
\end{tikzpicture}

\end{paracol}

\newpage

\subsection{Extreme Values}

\begin{paracol}{2}

\begin{tikzpicture}
\node [rounded-box] (box){\begin{minipage}{0.45\textwidth}
    Let $c$ be a number in the domain $D$ of a function $f$. Then $f(c)$ is the

    \begin{itemize}
        \item \textbf{absolute maximum} value of $f$ on $D$ if $f(c) \geq f(x)$
        \item \textbf{absolute minimum} value of $f$ on $D$ if $f(c) \leq f(x)$
    \end{itemize}
    
    for all $x$ in $D$.

    \begin{itemize}
        \item \textbf{local maximum} value of $f$ if $f(c) \geq f(x)$
        \item \textbf{local minimum} value of $f$ if $f(c) \leq f(x)$
    \end{itemize}

     when $x$ is near $c$.
\end{minipage}};
\node[rounded-box-title, left=10pt] at (box.north east) {Definition};
\end{tikzpicture}

\begin{tikzpicture}
\node [rounded-box] (box){\begin{minipage}{0.45\textwidth}
    If $f$ is continuous on a closed interval $[a, b]$, then $f$ attains an absolute maximum value $f(c)$ and an absolute minimum value $f(d)$ at some numbers $c$ and $d$ in $[a, b]$.
\end{minipage}};
\node[rounded-box-title, left=10pt] at (box.north east) {The Extreme Value Theorem};
\end{tikzpicture}

\begin{tikzpicture}
\node [rounded-box] (box){\begin{minipage}{0.45\textwidth}
    If $f$ has a local maximum or minimum at $c$, and if $f'(c)$ exists, then $f'(c) = 0$.
\end{minipage}};
\node[rounded-box-title, left=10pt] at (box.north east) {Fermat's Theorem};
\end{tikzpicture}

\switchcolumn

\begin{tikzpicture}
\node [rounded-box] (box){\begin{minipage}{0.45\textwidth}
    A \textbf{critical number} of a function $f$ is a number $c$ in the domain of $f$ s.t. either $f'(c) = 0$ or $f'(c)$ does not exist.
\end{minipage}};
\node[rounded-box-title, left=10pt] at (box.north east) {Definition};
\end{tikzpicture}

\begin{tikzpicture}
\node [rounded-box] (box){\begin{minipage}{0.45\textwidth}
    If $f$ has a local maximum or minimum at $c$, then $c$ is a critical number of $f$.
\end{minipage}};
\node[rounded-box-title, left=10pt] at (box.north east) {Corollary};
\end{tikzpicture}

\begin{tikzpicture}
\node [rounded-box] (box){\begin{minipage}{0.45\textwidth}
    To find the \textit{absolute} maximum and minimum values of a continuous function $f$ on a closed interval $[a, b]$: \\

    \begin{enumerate}
        \item Find the values of $f$ at the critical numbers of $f$ in $(a, b)$.
        \item Find the values of $f$ at the endpoints of the interval.
        \item The largest of the values from Steps 1 and 2 is the absolute maximum value; the smallest of these values is the absolute minimum value.
    \end{enumerate}
\end{minipage}};
\node[rounded-box-title, left=10pt] at (box.north east) {The Closed Interval Method};
\end{tikzpicture}

\end{paracol}

\subsection{The Mean Value Theorem}

\begin{paracol}{2}

\begin{tikzpicture}
\node [rounded-box] (box){\begin{minipage}{0.45\textwidth}
    Let $f$ be a function that satisfies the following three hypotheses: \\

    \begin{enumerate}
        \item $f$ is continuous on the closed interval $[a, b]$.
        \item $f$ is differentiable on the open interval $(a, b)$.
        \item $f(a) = f(b)$. \\
    \end{enumerate}

    Then there is a number $c$ in $(a, b)$ such that $f'(c) = 0$.
\end{minipage}};
\node[rounded-box-title, left=10pt] at (box.north east) {Rolle's Theorem};
\end{tikzpicture}

\begin{tikzpicture}
\node [rounded-box] (box){\begin{minipage}{0.45\textwidth}
    If $f'(x) = 0$ for all $x$ in an interval $(a, b)$, then $f$ is constant on $(a, b)$.
\end{minipage}};
\node[rounded-box-title, left=10pt] at (box.north east) {Theorem};
\end{tikzpicture}

\begin{tikzpicture}
\node [rounded-box] (box){\begin{minipage}{0.45\textwidth}
    If $f'(x) = g'(x)$ for all $x$ in an interval $(a, b)$, then $f - g$ is constant on $(a, b)$; that is, $f(x) = g(x) + c$ where $c$ is a constant.
\end{minipage}};
\node[rounded-box-title, left=10pt] at (box.north east) {Corollary};
\end{tikzpicture}

\switchcolumn

\begin{tikzpicture}
\node [rounded-box] (box){\begin{minipage}{0.45\textwidth}
    Let $f$ be a function that satisfies the following hypotheses:

    \begin{enumerate}
        \item $f$ is continuous on the closed interval $[a, ]$.
        \item $f$ is differentiable on the open interval $(a, b)$.
    \end{enumerate}

    Then there is a number $c$ in $(a, b)$ such that

    $$f'(c) = \frac{f(b) - f(a)}{b - a}$$

    or, equivalently,

    $$f(b) - f(a) = f'(c)(b - a)$$
\end{minipage}};
\node[rounded-box-title, left=10pt] at (box.north east) {The Mean Value Theorem};
\end{tikzpicture}

\begin{tikzpicture}
\node [rounded-box] (box){\begin{minipage}{0.45\textwidth}
    Suppose that the functions $f$ and $g$ are continuous on $[a, b]$ and differentiable on $(a, b)$, and $g'(x) \neq 0$ for all $x$ in $(a, b)$. Then there is a number $c$ in $(a, b)$ such that

    $$\frac{f'(c)}{g'(c)} = \frac{f(b) - f(a)}{g(b) - g(a)}$$
\end{minipage}};
\node[rounded-box-title, left=10pt] at (box.north east) {Cauchy's Mean Value Theorem};
\end{tikzpicture}

\end{paracol}

Note: Given the special case in which $g(x) = x$, then $g'(c) = 1$ and Cauchy's Mean Value Theorem is the Mean Value Theorem.

\subsection{Derivatives and the Shapes of Graphs}

\begin{paracol}{2}

\begin{tikzpicture}
\node [rounded-box] (box){\begin{minipage}{0.45\textwidth}
    \begin{itemize}
        \item If $f'(x) > 0$, then $f$ is increasing on that interval.
        \item If $f'(x) < 0$, then $f$ is decreasing on that interval.
    \end{itemize}
\end{minipage}};
\node[rounded-box-title, left=10pt] at (box.north east) {Increasing / Decreasing Test};
\end{tikzpicture}

\begin{tikzpicture}
\node [rounded-box] (box){\begin{minipage}{0.45\textwidth}
    Suppose $c$ is a critical number of a continuous function $f$.

    \begin{itemize}
        \item Local max: $f'$ changes from positive to negative.
        \item Local min: $f'$ changes from negative to positive.
        \item Neither: $f'$ does not change sign at $c$.
    \end{itemize}
\end{minipage}};
\node[rounded-box-title, left=10pt] at (box.north east) {The First Derivative Test};
\end{tikzpicture}

\begin{tikzpicture}
\node [rounded-box] (box){\begin{minipage}{0.45\textwidth}
    If the graph of $f$ lies above / below all of its tangents on an interval $I$, it is \textbf{concave upward / downward} on $I$.
\end{minipage}};
\node[rounded-box-title, left=10pt] at (box.north east) {Definition};
\end{tikzpicture}

\switchcolumn

\begin{tikzpicture}
\node [rounded-box] (box){\begin{minipage}{0.45\textwidth}
    \begin{itemize}
        \item Concave upward on $I$: $f''(x) > 0$ for all $x$ in $I$.
        \item Concave downward on $I$: $f''(x) < 0$ for all $x$ in $I$.
    \end{itemize}
\end{minipage}};
\node[rounded-box-title, left=10pt] at (box.north east) {Concavity Test};
\end{tikzpicture}

\begin{tikzpicture}
\node [rounded-box] (box){\begin{minipage}{0.45\textwidth}
    A point $P$ on a curve $y = f(x)$ is called an \textbf{inflection point} if $f$ is continuous there and the curve changes from concave upward to concave downward or from concave downward to concave upward at $P$.
\end{minipage}};
\node[rounded-box-title, left=10pt] at (box.north east) {Definition};
\end{tikzpicture}

\begin{tikzpicture}
\node [rounded-box] (box){\begin{minipage}{0.45\textwidth}
    \begin{itemize}
        \item Local minimum at $c$: $f'(c) = 0$ and $f''(c) > 0$.
        \item Local maximum at $c$: $f'(c) = 0$ and $f''(c) < 0$.
    \end{itemize}
\end{minipage}};
\node[rounded-box-title, left=10pt] at (box.north east) {The Second Derivative Test};
\end{tikzpicture}

\end{paracol}
