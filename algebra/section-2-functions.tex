\section{Functions}

\begin{paracol}{2}

\begin{tikzpicture}
\node [rounded-box] (box){\begin{minipage}{0.45\textwidth}
    Let $f: X \rightarrow Y$ be a function. The set $X$ is called the domain of $f$. The set $Y$ is called the co-domain of $f$.
\end{minipage}};
\node[rounded-box-title, left=10pt] at (box.north east) {Definition};
\end{tikzpicture}

\begin{tikzpicture}
\node [rounded-box] (box){\begin{minipage}{0.45\textwidth}
    Let $f: X \rightarrow Y$ be a function. \\

    \begin{itemize}
        \item $f$ is called \textbf{injective} or \textbf{one-to-one} if for all $a, b \in X$, if $f(a) = f(b)$ then $a = b$. \\

        \item The \textbf{image} of $f$, written $\text{im} \, f$, is $\{ f(x) : x \in X \}$. \\

        \item $f$ is called \textbf{surjective} or \textbf{onto} if $\text{im} \, f = Y$. \\

        \item $f$ is called a \textbf{bijection} if it is injective and surjective.
    \end{itemize}
\end{minipage}};
\node[rounded-box-title, left=10pt] at (box.north east) {Definition};
\end{tikzpicture}

\begin{tikzpicture}
\node [rounded-box] (box){\begin{minipage}{0.45\textwidth}
    Let $f: X \rightarrow Y$ and $g: Y \rightarrow Z$ be functions. The \textbf{composition} of $g$ and $f$, written $g \circ f$, is the function $g \circ f : X \rightarrow Z$ such that $(g \circ f)(x) = g(f(x))$.
\end{minipage}};
\node[rounded-box-title, left=10pt] at (box.north east) {Definition};
\end{tikzpicture}

NB: Composition only makes sense when the co-domain of $f$ is the same as the domain of $g$.

\begin{tikzpicture}
\node [rounded-box] (box){\begin{minipage}{0.45\textwidth}
    Function composition is associative. \\

    If $f: X \rightarrow Y, g: Y \rightarrow Z, h: Z \rightarrow W$, then

    $$h \circ (g \circ f) = (h \circ g) \circ f$$
\end{minipage}};
\node[rounded-box-title, left=10pt] at (box.north east) {Theorem};
\end{tikzpicture}

The reason this is true is because both sides send an input $x \in X$ to the output $h(g(f(x)))$.

\switchcolumn

\begin{tikzpicture}
\node [rounded-box] (box){\begin{minipage}{0.45\textwidth}
    The \textbf{identity function} $\text{id}_X$ does nothing: it is defined by $\text{id}_X(x) = x$ for all $x \in X$.
\end{minipage}};
\node[rounded-box-title, left=10pt] at (box.north east) {Definition};
\end{tikzpicture}

\begin{tikzpicture}
\node [rounded-box] (box){\begin{minipage}{0.45\textwidth}
    Let $f: X \rightarrow Y$ and $g: Y \rightarrow X$ be functions. Then \\

    \begin{itemize}
        \item $g$ is a \textbf{left inverse} to $f$, and $f$ is a \textbf{right inverse} to $g$, if $g \circ f = \text{im}_x$. \\

        \item $f$ is \textbf{invertible} if there is a function $h: Y \rightarrow X$ such that $f \circ h = \text{id}_Y$ and $h \circ f = \text{id}_X$. \\

        \item If $f$ is invertible, then there is one and only one function which is a left and right inverse to $f$ - its inverse $f^{-1}$.
    \end{itemize}
\end{minipage}};
\node[rounded-box-title, left=10pt] at (box.north east) {Definition};
\end{tikzpicture}

\begin{tikzpicture}
\node [rounded-box] (box){\begin{minipage}{0.45\textwidth}
    Let $f: X \rightarrow Y$ be a function. \\

    \begin{itemize}
        \item $f$ has a left inverse if and only if it is injective. \\

        \item $f$ has a right inverse if and only if it is surjective. \\

        \item $f$ is invertible if and only if it is a bijection.
    \end{itemize}
\end{minipage}};
\node[rounded-box-title, left=10pt] at (box.north east) {Theorem};
\end{tikzpicture}

\begin{tikzpicture}
\node [rounded-box] (box){\begin{minipage}{0.45\textwidth}
    If functions $f_1, f_2, \dots, f_n$ are invertible and the composition $f_1 \circ f_2 \circ \dots \circ f_n$ makes sense, then it is invertible with inverse $f_n^{-1} \circ f_{n-1}^{-1} \circ \dots \circ f_1^{-1}$.
\end{minipage}};
\node[rounded-box-title, left=10pt] at (box.north east) {Theorem};
\end{tikzpicture}

\end{paracol}
