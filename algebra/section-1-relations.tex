\section{Relations}

\begin{paracol}{2}

\begin{tikzpicture}
\node [rounded-box] (box){\begin{minipage}{0.45\textwidth}
    A \textbf{relation} on a set $X$ is a property of an ordered pair of elements of $X$ which can be true or false.
\end{minipage}};
\node[rounded-box-title, left=10pt] at (box.north east) {Definition};
\end{tikzpicture}

\textbf{Example}: $<$ is a relation on the set of natural numbers: if $a$ and $b$ are natural numbers then $a < b$ is either true or false.

\begin{tikzpicture}
\node [rounded-box] (box){\begin{minipage}{0.45\textwidth}
    \textbf{Properties of relations}: Let $\sim$ be a relation on a set $X$. \\

    \begin{itemize}
        \item $\sim$ is called \textbf{symmetric} if for any $x, y \in X$ if $x \sim y$ then $y \sim x$. \\

        \item $\sim$ is called \textbf{reflexive} if for any $x \in X$ we have $x \sim x$. \\

        \item $\sim$ is called \textbf{transitive} if for any $x, y, z \in X$ if $x \sim y$ and $y \sim z$ then $x \sim z$. \\

        \item $\sim$ is called an \textbf{equivalence relation} if it is reflexive, symmetric and transitive.
    \end{itemize}
\end{minipage}};
\node[rounded-box-title, left=10pt] at (box.north east) {Definition};
\end{tikzpicture}

\switchcolumn

\begin{tikzpicture}
\node [rounded-box] (box){\begin{minipage}{0.45\textwidth}
    Let $\sim$ be an equivalence relation on a set $X$, and let $x \in X$. The \textbf{equivalence class} of $x$, written $[x]$ or $[x]_\sim$, is

    $$[x] = \{ y \in X | y \sim x \}$$
\end{minipage}};
\node[rounded-box-title, left=10pt] at (box.north east) {Definition};
\end{tikzpicture}

\begin{tikzpicture}
\node [rounded-box] (box){\begin{minipage}{0.45\textwidth}
    Let $\sim$ be an equivalence relation on a set $X$. Then \\

    \begin{itemize}
        \item Every $x \in X$ belongs to some equivalence class. \\

        \item If two equivalence classes classes are not disjoint, then they are equal.
    \end{itemize}
\end{minipage}};
\node[rounded-box-title, left=10pt] at (box.north east) {Theorem};
\end{tikzpicture}

\end{paracol}
