\section{Permutations}

\begin{paracol}{2}

\begin{tikzpicture}
\node [rounded-box] (box){\begin{minipage}{0.45\textwidth}
    A \textbf{permutation} of a set $X$ is a bijection from $X$ to $X$. \\

    For a set $X = \{1, 2, \dots, n\}$, the set of all permutations on $X$ is called the \textbf{symmetric group on n letters}, $S_n$.
\end{minipage}};
\node[rounded-box-title, left=10pt] at (box.north east) {Definition};
\end{tikzpicture}

\switchcolumn

\begin{tikzpicture}
\node [rounded-box] (box){\begin{minipage}{0.45\textwidth}
    If $\sigma$ and $\tau$ are permutations, we will often write their \textbf{composition} $\sigma \circ \tau$ as $\sigma \tau$, and refer to it as the \textbf{product} of $\sigma$ and $\tau$.
\end{minipage}};
\node[rounded-box-title, left=10pt] at (box.north east) {Definition};
\end{tikzpicture}

\end{paracol}

\subsection{Two-row notation}

\begin{paracol}{2}

\begin{tikzpicture}
\node [rounded-box] (box){\begin{minipage}{0.45\textwidth}
    Given a permutation

    $$
    \sigma = \begin{pmatrix}
        1 & 2 & 3 & 4 & 5 \\
        3 & 4 & 5 & 1 & 2
    \end{pmatrix}
    $$

    then on swapping the rows gives

    $$
    \sigma^{-1} = \begin{pmatrix}
        3 & 4 & 5 & 1 & 2 \\
        1 & 2 & 3 & 4 & 5
    \end{pmatrix}
    $$

    and rearranging gives the \textbf{inverse} permutation

    $$
    \sigma^{-1} = \begin{pmatrix}
        1 & 2 & 3 & 4 & 5 \\
        4 & 5 & 1 & 2 & 3
    \end{pmatrix}
    $$
\end{minipage}};
\node[rounded-box-title, left=10pt] at (box.north east) {Definition};
\end{tikzpicture}

\switchcolumn

\begin{tikzpicture}
\node [rounded-box] (box){\begin{minipage}{0.45\textwidth}
    $$\mid S_n \mid = n!$$
\end{minipage}};
\node[rounded-box-title, left=10pt] at (box.north east) {Theorem};
\end{tikzpicture}

\textbf{Proof}: Induction on $n$.

When $n = 1$ there is a unique bijection $\{1\} \rightarrow \{1\}$, namely the identity map, so $\mid S_1 \mid = 1 = 1!$ as required.

The number of elements of $S_n$ is the number of different ways to order the elements $1, 2, \dots, n$. An ordering of $1, 2, \dots, n$ is the same thing as an ordering of $1, 2, \dots, n-1$ with $n$ inserted into one of $n$ positions, so the number of possible orderings is $n$ times the number of orderings of $1, \dots, n-1$, which is $(n-1)!$ by the inductive hypothesis.

So $\mid S_n \mid = x \times (n-1)! = n!$.

\end{paracol}

\subsection{Cycles}

\begin{paracol}{2}

\begin{tikzpicture}
\node [rounded-box] (box){\begin{minipage}{0.45\textwidth}
    Let $a_0, \dots, a_{m-1}$ be distinct elements of $\{1, 2, \dots, n\}$. Then $(a_0, \dots, a_{m-1})$ is the permutation in $S_n$ such that

    \begin{itemize}
        \item $a_i \mapsto a_{i+1}$ for $0 \leq i \leq m-1$, $a_{m-1} \mapsto a_0$,
        \item and if $x \neq a_1, \dots, a_m$ then $x \mapsto x$.
    \end{itemize}

    Such a permutation is called an $m$-\textbf{cycle}.
\end{minipage}};
\node[rounded-box-title, left=10pt] at (box.north east) {Definition};
\end{tikzpicture}

\begin{tikzpicture}
\node [rounded-box] (box){\begin{minipage}{0.45\textwidth}
    A permutation which is an $m$-cycle for some $m$ is called a cycle.
\end{minipage}};
\node[rounded-box-title, left=10pt] at (box.north east) {Definition};
\end{tikzpicture}

\textbf{Counter-example}: Not every permutation is a cycle, e.g. $\begin{pmatrix}
    1 & 2 & 3 & 4 \\
    2 & 1 & 4 & 3
\end{pmatrix}$.

\begin{tikzpicture}
\node [rounded-box] (box){\begin{minipage}{0.45\textwidth}
    Two cycles $(a_0, \dots, a_{m-1})$ and $(b_0, \dots, b_{m-1})$ are \textbf{disjoint} if no $a_i$ is equal to any $b_j$.
\end{minipage}};
\node[rounded-box-title, left=10pt] at (box.north east) {Definition};
\end{tikzpicture}

Any permutation can be written as a product of disjoint cycles, e.g. the permutation above is equal to $(1, 2)(3, 4)$.

\begin{tikzpicture}
\node [rounded-box] (box){\begin{minipage}{0.45\textwidth}
    Let $m \in \mathbb{Z}$ and $\sigma \in S_n$. Then

    \begin{equation*}
        \sigma^m = \begin{cases}
            \sigma \circ \dots \circ \sigma (m \text{ times}) & m > 0 \\
            \text{id} & m = 0 \\
            \sigma^{-1} \circ \dots \circ \sigma^{-1} (-m \text{ times}) & m < 0 \\
        \end{cases}
    \end{equation*}

    and for any $a, b \in \mathbb{Z}$,

    $$\sigma^a \sigma^b = \sigma^{a+b}$$
\end{minipage}};
\node[rounded-box-title, left=10pt] at (box.north east) {Definition | Theorem};
\end{tikzpicture}

\switchcolumn

\begin{tikzpicture}
\node [rounded-box] (box){\begin{minipage}{0.45\textwidth}
    \textbf{Description}: Text
    $$\mathbf{v} \cdot \mathbf{w} = 0 \iff \alpha = \frac{\pi}{2} \iff \mathbf{v} \perp \mathbf{w}$$
\end{minipage}};
\node[rounded-box-title, left=10pt] at (box.north east) {Definition | Theorem};
\end{tikzpicture}

\begin{tikzpicture}
\node [rounded-box] (box){\begin{minipage}{0.45\textwidth}
    \textbf{Description}: Text
    $$\mathbf{v} \cdot \mathbf{w} = 0 \iff \alpha = \frac{\pi}{2} \iff \mathbf{v} \perp \mathbf{w}$$
\end{minipage}};
\node[rounded-box-title, left=10pt] at (box.north east) {Definition | Theorem};
\end{tikzpicture}

\end{paracol}
