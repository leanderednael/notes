\section{Groups}

A group is a very simple mathematical object consisting of two things: (a) a \textbf{set} $G$ and (b) a way of combining two elements of the set to produce another, called the \textbf{group operation}.

This group operation has to obey three rules mimicing those obeyed by the symmetries of a physical object called the group axioms.

\begin{tikzpicture}
\node [rounded-box] (box){\begin{minipage}{0.45\textwidth}
    A group $(G, *)$ is a set $G$ with a binary operation $*$ which contains an element $e$ such that \\

    \begin{itemize}
        \item (Identity axiom) For all $g \in G$, $e * g = g * e = g$. \\

        \item (Inverses axiom) For all $g \in G$, there exists $h \in G$ such that $h * g = g * h = e$. \\

        \item (Associativity axiom) For all $g, h, k \in G$, $(g * h) * k = g * (h * k)$.
    \end{itemize}
\end{minipage}};
\node[rounded-box-title, left=10pt] at (box.north east) {Definition};
\end{tikzpicture}

A binary operation on $G$ is a function that takes as input a pair of elements of $G$ and outputs a single element of $G$: that is, a function $G \times G \rightarrow G$.

\textbf{Examples}:

\begin{itemize}
    \item $+$ is a binary operation on the set of integers $\mathbb{Z}$.
    \item $-$ is a binary operation on the set of complex numbers $\mathbb{C}$.
    \item $-$ is \textbf{not} a binary operation on the set of strictly positive integers $\mathbb{N}$, because it doesn't always output an element of $\mathbb{N}$.
    \item $a * b = 2 \quad \forall \quad a, b \in \mathbb{R}$ is a binary operation on the real numbers $\mathbb{R}$.
\end{itemize}

\begin{tikzpicture}
\node [rounded-box] (box){\begin{minipage}{0.45\textwidth}
    \textbf{Description}: Text
    $$\mathbf{v} \cdot \mathbf{w} = 0 \iff \alpha = \frac{\pi}{2} \iff \mathbf{v} \perp \mathbf{w}$$
\end{minipage}};
\node[rounded-box-title, left=10pt] at (box.north east) {Theorem};
\end{tikzpicture}

\begin{tikzpicture}
\node [rounded-box] (box){\begin{minipage}{0.45\textwidth}
    \textbf{Description}: Text
    $$\mathbf{v} \cdot \mathbf{w} = 0 \iff \alpha = \frac{\pi}{2} \iff \mathbf{v} \perp \mathbf{w}$$
\end{minipage}};
\node[rounded-box-title, left=10pt] at (box.north east) {Definition | Theorem};
\end{tikzpicture}

\subsection{The Symmetric Group}

\begin{tikzpicture}
\node [rounded-box] (box){\begin{minipage}{0.45\textwidth}
    \textbf{Description}: Text
    $$\mathbf{v} \cdot \mathbf{w} = 0 \iff \alpha = \frac{\pi}{2} \iff \mathbf{v} \perp \mathbf{w}$$
\end{minipage}};
\node[rounded-box-title, left=10pt] at (box.north east) {Definition | Theorem};
\end{tikzpicture}

\subsection{Subgroups}

\begin{tikzpicture}
\node [rounded-box] (box){\begin{minipage}{0.45\textwidth}
    \textbf{Description}: Text
    $$\mathbf{v} \cdot \mathbf{w} = 0 \iff \alpha = \frac{\pi}{2} \iff \mathbf{v} \perp \mathbf{w}$$
\end{minipage}};
\node[rounded-box-title, left=10pt] at (box.north east) {Definition | Theorem};
\end{tikzpicture}

\subsection{Cosets and Lagrange's Theorem}

\begin{tikzpicture}
\node [rounded-box] (box){\begin{minipage}{0.45\textwidth}
    \textbf{Description}: Text
    $$\mathbf{v} \cdot \mathbf{w} = 0 \iff \alpha = \frac{\pi}{2} \iff \mathbf{v} \perp \mathbf{w}$$
\end{minipage}};
\node[rounded-box-title, left=10pt] at (box.north east) {Definition | Theorem};
\end{tikzpicture}

\subsection{The Dihedral Groups}

\begin{tikzpicture}
\node [rounded-box] (box){\begin{minipage}{0.45\textwidth}
    \textbf{Description}: Text
    $$\mathbf{v} \cdot \mathbf{w} = 0 \iff \alpha = \frac{\pi}{2} \iff \mathbf{v} \perp \mathbf{w}$$
\end{minipage}};
\node[rounded-box-title, left=10pt] at (box.north east) {Definition | Theorem};
\end{tikzpicture}

\subsection{Homomorphisms and Isomorphisms}

\begin{tikzpicture}
\node [rounded-box] (box){\begin{minipage}{0.45\textwidth}
    \textbf{Description}: Text
    $$\mathbf{v} \cdot \mathbf{w} = 0 \iff \alpha = \frac{\pi}{2} \iff \mathbf{v} \perp \mathbf{w}$$
\end{minipage}};
\node[rounded-box-title, left=10pt] at (box.north east) {Definition | Theorem};
\end{tikzpicture}
